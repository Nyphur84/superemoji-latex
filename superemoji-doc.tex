\documentclass{article}

\usepackage{fontspec}
\usepackage{etoolbox}       % for \ifcsdef
\usepackage{superemoji}

% Use an emoji-capable font (adjust if needed)
\renewfontfamily\EmojiFont{Segoe UI Emoji}

% Make \emoji use the emoji font and the semantic mappings from superemoji.sty
\renewcommand{\emoji}[1]{%
	\ifcsdef{emoji@#1}%
	{{\EmojiFont \csname emoji@#1\endcsname}}%
	{?}% fallback for unknown keys
}

\title{The \texttt{superemoji} Package}
\author{Kai Günther}
\date{December 8, 2025}

\begin{document}
	\maketitle
	
	\section{Overview}
	
	The \texttt{superemoji} package provides semantic emoji commands such as
	\verb|\emoji{status-ok1}| instead of raw Unicode emoji in the source.
	Each semantic key (for example \verb|status-ok1| or \verb|emo-joy1|)
	is mapped to one or more Unicode emoji in an external JSON file that is
	compiled into \LaTeX{} macros by the package.
	
	\section{Loading the package}
	
	At minimum, load \texttt{fontspec} and \texttt{superemoji} and select an
	emoji-capable font. The document should be compiled with Lua\LaTeX{} or
	Xe\LaTeX{} so that Unicode emoji are supported.
	
	\begin{verbatim}
		\usepackage{fontspec}
		\usepackage{superemoji}
		
		\renewfontfamily\EmojiFont{Segoe UI Emoji}
		
		\renewcommand{\emoji}[1]{%
			\ifcsdef{emoji@#1}%
			{{\EmojiFont \csname emoji@#1\endcsname}}%
			{?}%
		}
	\end{verbatim}
	
	The macro \verb|\emoji{<key>}| looks up a control sequence
	\verb|\emoji@<key>| defined by the package and typesets it in the
	\verb|\EmojiFont|. If an unknown key is requested, a simple \verb|?| is
	shown as a fallback.
	
	\section{Semantic keys and subsets}
	
	Keys are organized by purpose. Examples:
	
	\begin{itemize}
		\item \textbf{Status and logs:}
		\verb|status-ok1|, \verb|status-error1|,
		\verb|log-debug1|, \verb|log-run1|.
		\item \textbf{GIS workflows:}
		\verb|gis-select1|, \verb|gis-buffer1|,
		\verb|gis-snap1|.
		\item \textbf{Navigation and maps:}
		\verb|map-right1|, \verb|map-pin1|.
		\item \textbf{Emotions (\texttt{emo} subset):}
		\verb|emo-joy1|, \verb|emo-sad1|,
		\verb|emo-angry1|, \verb|emo-deadinside1|.
		\item \textbf{Flags:}
		\verb|flag-de|, \verb|flag-eu|, etc.
	\end{itemize}
	
	Using semantic keys keeps the \LaTeX{} source readable and stable even if
	the underlying emoji choices change later.
	
	\section{Quick start}
	
	A minimal quick-start snippet:
	
	\begin{verbatim}
		\usepackage{superemoji}
		Status: \emoji{status-ok1}
	\end{verbatim}
	
	Rendered in this document, the same commands look like:
	
	\medskip
	\noindent
	Status: \emoji{status-ok1}
	
	\noindent
	Debug: \emoji{log-debug1}\quad
	GIS Select: \emoji{gis-select1}\quad
	Happy: \emoji{emo-joy1}\quad
	Flag: \emoji{flag-de}.
	
	\section{Minimal example}
	
	The following minimal document shows how to load the package and use a
	few semantic emoji commands. Compile with Lua\LaTeX{} or Xe\LaTeX{}.
	
	\begin{verbatim}
		\documentclass{article}
		\usepackage{fontspec}      % for Unicode emoji
		\usepackage{etoolbox}      % for \ifcsdef
		\usepackage{superemoji}    % this package
		
		\renewfontfamily\EmojiFont{Segoe UI Emoji}
		
		\renewcommand{\emoji}[1]{%
			\ifcsdef{emoji@#1}%
			{{\EmojiFont \csname emoji@#1\endcsname}}%
			{?}%
		}
		
		\begin{document}
			
			Status: \emoji{status-ok1}
			
			Debug:  \emoji{log-debug1}
			GIS:    \emoji{gis-select1}
			
			Mood:   \emoji{emo-joy1}, \emoji{emo-sad1}, \emoji{emo-deadinside1}
			
		\end{document}
	\end{verbatim}
	
	When compiled, this produces:
	
	\medskip
	\noindent
	Status: \emoji{status-ok1}
	
	\noindent
	Debug: \emoji{log-debug1}\quad
	GIS: \emoji{gis-select1}
	
	\noindent
	Mood: \emoji{emo-joy1}, \emoji{emo-sad1}, \emoji{emo-deadinside1}
	
\end{document}
